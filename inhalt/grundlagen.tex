\chapter{Grundlagen/Fundamentals}
\section{Kryptographie}

Kryptographie die Lehre der Absicherung von Narchichten durch verschluesselung \cite[18]{crypto}.
Verschluesselung oder Chifferierung bezeichnet das Verfahren, um eine Nachricht unverstaendlich zu machen \cite[18]{crypto}.
Die sicherheit aller kryptographischen Verfahren basiert im Wesentlichen auf der Schwierigkeit, einen geheimen Schluessel zu erraten oder ihn auf anderem Wege zu beschaffen.
Es ist durchaus moeglich, einen Schluessel zu erraten, wenn auch die Wahrscheinlichkeit mit wachsender Schluessellaenge sehr klein wird. Absolute sicherheit gibt es in der Kryptographie nicht.
\cite[25]{crypto}.

Praktische alle kryptographischen Verfahren haben die Aufgabe, eine der folgenden vier Eigenschaften von Nachrichten zu gewaehrleisten. \cite[18]{crypto}
\begin{itemize}
    \item[Geheimhaltung:] Ziel der Geheimhaltung ist es, das Lesen einer Narchicht fuer Unbefugte unmoeglich bzw. schwierig zu machen.
    \item[Authentifizierung] oder Authentifikation: Identitaetsbeweis des Senders einer Narchicht gegenueber dem Empfaenger, d.h. der Empfaenger kann sicher sein, dass die Narchicht nicht von einem anderen (unbefugten) Absender stammt.
    \item[Integritaet] Die Nachricht darf waehrend der Uebermittlung nicht (von Unbefugten) veraendert werden. Sie bewahrt ihre Integritaet, das heisst ihre Unverletztheit.
    \item[Verbindlichkeit] Der Sender kann spaeter nicht leugnen, eine Nachricht abgeschickt zu haben.
\end{itemize}

Kryptographische Algorithmen sind Berechnungsvorschriften, d.h. mathematische Funktionen zur Ver-und Entschluesselung \cite[19]{crypto}.
Ein kryptographischer algorithmus zum verschluesseln kann auf vielfaeltige Art und Weise in unterschiedlichen Anwendungen eingesetzt werden.\cite[22]{eg-spec}.
Damit eine Anwendung immer in der gleichen und korrekten
Art ablauft, werden kryptographische Protokolle definiert. Im Gegensatzt zu den krzptographischen Algorithmen handelt es sich bei den Protokollen um Verfahren zur Steuerung des Ablaufs von Transaktionen fuer bestimmte Anwendungen. \cite[22]{eg-spec}.

Es gibt viele verschiedene Realisierungen und Vorschlaege fuer elektroinsiche Wahlen. Ein besonders elegantes (und noch relativ neues) Protokoll wird
hier vorgestellt. Zuvor muessen jedoch noch zwei Hilfsmittel bereitgestellt werden.

\subsection{Elgamal}
Der erste 1976 publizierte, public key-algorithmus von Diffie und Hellman. Er dient dem Schluesselaustausch, kann aber nicht zum Signieren verwendet werden.
Eine Verallgemeinerung, die auch zum Verschluesselung und zum Signieren taugt, ist der Algorithmus von ElGamal.\cite[78]{crypto}.


\cite[78]{crypto}

Der ElGamal-Algorithmus ist eine Verallgemeinerung des Diffie-Hellman-Algorithmus. \cite[95]{crypto}.
Diffie-Hellman ist ein Public-Key Algorithmus \cite[94]{crypto}. Mit Hilfe der Public-Key-Kryptography kann nun jedermann mit beliebigen Partnern geheime Narchichten austauschen, Dokumente signieren und viele andere kryptographische Anwendungen nutzen.\cite[22]{crypto}

\subsubsection{Einweg-Hash-Funktionen}
Einwegfunktionen sind funktionen die sich leicht berechnen lassen, deren Umkehrung jedoch nicht oder nur sehr schwer (
d.h. mit
nicht
vertretbaren Aufwand) zu berechnen ist, insebsondere, wenn die funktion oeffentlich bekannt ist. \cite[100]{crypto}.
Einweg-Hash-Funktionen sind Einwegfunktionen, die beliebig lange Klartexte auf einen Hash-Wert fester Laenge abbilden.\
Secure-Hash-Algorithmus (SHA)

\subsection{Non-interactive zero-knowledge (NIZK)}
Non-interactive zero-knowledge (NIZK) proofs ElectionGuard provides numerous proofs about encryption keys, encrypted
ballots, and election tallies using the following four techniques \cite[6]{eg-spec}
\subsubsection{Schnorr proof}
1. A Schnorr proof7 allows the holder of an ElGamal
secret key s to interactively prove possession of s without revealing s. \cite[6]{eg-spec}.
\subsubsection{Chaum-Pedersen}
2. A Chaum-Pedersen proof8 allows an ElGamal
encryption to be interactively proven to decrypt to a particular value without revealing the nonce used for encryption
or the secret decryption key s. (This proof can be constructed with access to either the nonce used for encryption or
the secret decryption key.) \cite[6]{eg-spec}
\subsubsection{Cramer-Damg˚ ard-Schoenmakers technique}
3. The Cramer-Damg˚ ard-Schoenmakers technique9 enables a disjunction to be interactively
proven without revealing which disjunct is true.
\subsubsection{Fiat-Shamir heuristic}
4. The Fiat-Shamir heuristic10 allows interactive proofs to be
converted into non-interactive proofs. \cite[6]{eg-spec}

Using a combination of the above techniques, it is possible for ElectionGuard to
demonstrate that keys are properly chosen, that ballots are properly formed, and that decryptions match claimed values.
\cite[6]{eg-spec}

Threshold Encryption
Additive Homomorphic encryption


Einleitung Wahlen und vorstellung von ElectionGuard

\section{ElectionGuard}



ElectionGuard is not a complete election system. It instead provides components for system developers to implement
E2E-verifiable elections. The goal of ElectionGuard is to promote voter confidence by empowering voters to
indepently verify the accuracy of election results. \cite[1]{eg-spec}

An E2E-verifiable election provides artifacts which allow voters to confirm that their votes have been accurately
recorded and counted. Specifically, an election is End-to-end (E2E) verifiable if two properties are achieved. 1.
Individual voters can verify that their votes have been accurately recorded. 2. Voters and observers can verify that all
recorded votes have been accurately counted.\cite[1]{eg-spec}

ElectionGuard achieves this by

\subsection{Verifier Construction}
the signature. The entire election record and its digital signature should be
\cite[25]{eg-spec}
published and made available for full download by any interested individuals. Tools should also be provided for easy
look up of tracking codes by voters.
\cite[25]{eg-spec}

\subsection{ElectionGuard Components}
Four Principal components of ElectionGuard are described below \cite[3-4]{eg-spec}
\begin{itemize}
    \item[Parameter Requirements] These are properties required of parameters that are standard in every election. A
    specific set of standard parameters is provided [...].
    \item[Key Generation] Prior to each individual election, guardians must generate individual publicprivate key
    pairs and exchange shares of private keys to enable completion of an election even if some guardians become
    unavailable.
    \item[Ballot Encryption] While encrypting the contents of a ballot is a relatively simple operation, most of the
    work of ElectionGuard is the process of creating externally-verifiable artifacts to prove that each
    encrypted ballot is well-formed (i.e., its decryption is a legitimate ballot without overvotes or improper values).
    \item[Verifiable Decryption]
    At the conclusion of each election, guardians use their private keys to produce election tallies
    together with verifiable artifacts that prove that the tallies are correct.
\end{itemize}

\subsubsection{Parameter Requirements}

\subsubsection{Key Generation}
Prior to the start of voting, trustees participate in a process wherein they generate public keys to be used in the
election. These trustees are called "Guardians" in ElectionGuard. Members of a canvassing board could serve as
guardians \cite[2]{eg-spec}.

Each guardian generates its own public-private key pair. These public keys will be combined to form a single public
key which will be used to encrypt individual ballots. They are also used individually by guardians to exchange
information about their private keys so that the election record can be produced after voting or auditing is
complete – even if not all guardians are available at that time.


The key generation ceremony begins with each guardian publishing its public keys together with proofs of knowledge
of the associated private keys. Once all public keys are published, each guardian uses each other guardian’s public
auxiliary key to encrypt shares of its own private keys. Finally, each guardian decrypts the shares it receives from
other guardians and checks them for consistency. If the received shares verify, the receiving guardian announces its
completion. If any shares fail to verify, the receiving guardian challenges the sender. In this case, the sender is
obliged to reveal the shares it sent. If it does so and the shares verify, the ceremony concludes and the election
proceeds. If a challenged guardian fails to produce key shares that verify, that guardian is removed and the key
generation ceremony restarts with a replacement guardian. \cite[2]{eg-spec}

Although it is preferred to generate new keys for each election, it is permissible to use the same
keys for multiple elections so long as the set of guardians remains the same. A complete new set of keys must
be generated if even a single guardian is replaced.\cite[3]{eg-spec}

5As will be seen below, the actual public key used to encrypt votes will be a combination of separately generated public
keys. So, no entity will ever be in possession of a private key that can be used to decrypt
-\cite[5]{eg-spec}

3.2 Key Generation Before an election, the number of guardians (n) is fixed together with a quorum value (k) that
describes the number of guardians necessary to decrypt tallies and produce election verification data. The values n and
k are integers subject to the constraint that 1 ≤ k ≤ n. Canvassing board members can often serve the role of election
guardians, and typical values for n and k could be 5 and 3 – indicating that 3 of 5 canvassing board members must
cooperate to produce the artifacts that enable election verification. The reason for not setting the quorum value k too
low is that it will also be possible for k guardians to decrypt individual ballots.
\cite[8]{eg-spec}

\subsubsection{Ballot Encryption}
Ballot encryption In most uses, the election system makes a single call to the ElectionGuard API after each voter
completes the process of making selections [...]. ElectionGuard will encrypt
the selections made by the voter and return a verification code which the system should give to the voter.1
\cite[3]{eg-spec}

The encrypted ballots are published along with non-interactive zero-knowledge proofs of their integrity. The encryption
method used herein has a homomorphic property which allows the encrypted ballots to be combined into a single aggregate
ballot which consists of encryptions of the election tallies.
\cite[3]{eg-spec}

Encryption of votes in ElectionGuard is performed using an exponential form of the ElGamal cryptosystem4.
\cite[4]{eg-spec}.

An encryption of one is used to indicate that an option is selected, and an encryption of zero is used to indicate that
an option is not selected.\cite[5]{eg-spec}

Homomorphic properties A fundamental quality of the exponential form of ElGamal described above is its additively
homomorphic property. If two messages M1 and M2 are respectively encrypted as (A1, B1) = (gR1 mod p, gM1 · KR1 mod p)
and (A2, B2) = (gR2 mod p, gM2 · KR2 mod p), then the componentwise product (A, B) = (A1A2 mod p, B1B2 mod p) = (gR1+R2
mod p, gM1+M2 · KR1+R2 mod p) is an encryption of the sum M1 + M2.
\cite[5]{eg-spec}.

This additively homomorphic property is used in two important ways in ElectionGuard. First, all of the encryptions of a
single option across ballots can be multiplied to form an encryption of the sum of the individual values. Since the
individual values are one on ballots that select that option and zero otherwise, the sum is the tally of votes for that
option and the product of the individual encryptions is an encryption of the tally. The other use is to sum all of the
selections made in a single contest on a single ballot. After demonstrating that each option is an encryption of either
zero or one, the product of the encryptions indicates the number of options that are encryptions of one, and this can be
used to show that no more ones than permitted are among the encrypted options – i.e., that no more options were selected
than permitted. However, as will be described below, it is possible for a holder of a nonce R to prove to a third party
that a pair (α, β) is an encryption of M without revealing the nonce R and without access to the secret s.
\cite[5]{eg-spec}

Encryption of other data ElectionGuard provides means to encrypt data other than votes, which are selections encoded as
zero or one that need to be homomorphically aggregated. Such data may include the cryptographic shares of a guardian’s
private key that are sent to the other guardians and are encrypted to the receiving guardians’ public keys, write-in
information that needs to be attached to an encrypted selection and is encrypted to the election public key, or other
auxiliary data attached to an encrypted ballot, either encrypted to the election public key or to an administrative
public key. The non-vote data do not need to be homomorphically encrypted and can use a more standard form of public-key
encryption removing the data size restrictions imposed by the exponential ElGamal scheme for vote encryption.
ElectionGuard encrypts such data with hashed ElGamal encryption, which deploys a key derivation function (KDF) to
generate a key stream that is then XORed with the data. To implement the KDF and to provide a message authentication
code (MAC), encryption makes use of the keyed Hash Message Authentication Code HMAC. In ElectionGuard, HMAC is
instantiated as HMAC-SHA-256 with the hash function SHA-256.
\cite[7]{eg-spec}

\subsubsection{Verifiable Decryption}
After the conclusion of voting or
auditing, a quorum of guardians is necessary to produce the artifacts required to enable public verification of the
tally. \cite[2]{eg-spec}

Verifiable decryption
In the final step, election guardians independently use their secret keys to decrypt the election tallies and associated
verification data. It is not necessary for all guardians to be available to complete this step. If some guardians are
missing, a quorum of guardians can use the previously shared key fragments to reconstruct the missing verification data.
\cite[3]{eg-spec}

Observers can use this open specification and/or accompanying materials to write election verifiers that can confirm the
integrity of each encrypted ballot, the correct aggregation of these ballots, and the accurate decryption of election
tallies.
\cite[3]{eg-spec}


Threshold encryption Threshold ElGamal encryption is used for encryption of ballots and other data. This form of
encryption makes it very easy to combine individual guardian public keys into a single public key. It also offers a
homomorphic property that allows individual encrypted votes to be combined to form encrypted tallies. The guardians of
an election will each generate a public-private key pair. The public keys will then be combined (as described in the
following section) into a single election public key which is used to encrypt all selections made by voters in the
election. Ideally, at the conclusion of the election, each guardian will use its private key to form a verifiable
partial decryption of each tally. These partial decryptions will then be combined to form full verifiable decryptions of
the election tallies. To accommodate the possibility that one or more of the guardians will not be available at the
conclusion of the election to form their partial decryptions, the guardians will
\cite[6]{eg-spec}

\subsection{Parameter Requirements}
share11 their private keys amongst each other during key generation in a manner to be detailed in the next section. A
pre-determined threshold quorum value (k) out of the (n) guardians will be necessary to produce a full decryption.
\cite[7]{eg-spec}



Parameter Requirements Specific values for primes p and q and a generator g are given in Section 4 Baseline Parameters
below.
\cite[7]{eg-spec}

\subsection{Key Generation}
of an election is a public ballot coding file. This file lists all the contests in an election, the number of selections
allowed for each contest, and the options for each contest together with associations between each option and its
representation on a virtual ballot. It is assumed that each contest in the ballot coding file has a unique label and
that within each contest, each option also has a unique label. For
\cite[8]{eg-spec}

But the contents of this file are hashed together with the prime modulus (p), subgroup order (q), generator (g), number
of guardians (n), decryption quorum threshold value (k), date, and jurisdictional information to form a base hash code (
Q) which will be incorporated into every subsequent hash computation in the election.
\cite[8]{eg-spec}



Note that decryption of individual ballots does not directly compromise voter privacy since links between encrypted
ballots and the voters who cast them are not retained by the system. However, voters receive verification codes that can
be associated with individual encrypted ballots, so any group that has the ability to decrypt individual ballots can
also coerce voters by demanding to see their tracking codes. Threshold ElGamal encryption is used for encryption of
ballots. This form of encryption makes it very easy to combine individual guardian public keys into a single public key
for encrypting votes and ballots. It also offers a homomorphic property that allows individual encrypted votes to be
combined to form encrypted tallies. The guardians of an election will each generate a public-private key pair. The
public keys will then be combined (as described in the following section) into a single election public key which is
used to encrypt all selections made by voters in the election. Ideally, at the conclusion of the election, each guardian
will use its private key to form a verifiable partial decryption of each tally. These partial decryptions will then be
combined to form full verifiable decryptions of the election tallies. To accommodate the possibility that one or more of
the guardians will not be available at the conclusion of the election to form their partial decryptions, the guardians
will cryptographically share13 their private keys amongst each other during key generation in a manner to be detailed in
the next section. A pre-determined threshold value (k) out of the (n) guardians will be necessary to produce a full
decryption.
\cite[8]{eg-spec}

\subsection{Ballot Encryption}
An ElectionGuard ballot is comprised entirely of encryptions of one (indicating selection made) and zero (indicating
selection not made). To enable homomorphic addition (for tallying), these values are exponentiated during encryption.
\cite[12]{eg-spec}

A contest in an election consists of a set of options together with a selection limit that indicates the number of
selections that are allowed to be made in that contest. In most elections, most contests have a selection limit
\cite[12]{eg-spec}

A legitimate vote in a contest consists of a set of selections with cardinality not exceeding the selection limit of
that contest. To accommodate legitimate undervotes, the internal representation of a contest is augmented with
“placeholder” options equal in number to the selection limit. Placeholder options are selected as necessary to force the
total number of selections made in a contest to be equal to the selection limit. When the selection limit is one, for
example, the single placeholder option can be thought of as a “none of the above” option. With larger selection limits,
the number of placeholder options selected corresponds to the number of additional options that a voter could have
selected
\cite[12]{eg-spec}
options selected corresponds to the number of additional options that a voter could have selected in a contest. For
efficiency, the placeholder options could be eliminated in an approval vote. However, to
\cite[12]{eg-spec}

Two things must now be proven about the encryption of each vote to ensure ballot integrity. 1. The encryption associated
with each option is either an encryption of zero or an encryption of one. 2. The sum of all encrypted values in each
contest is equal to the selection limit for that contest (usually one).
\cite[12]{eg-spec}

\subsection{Outline for proofs of ballot correctness}

The use of ElGamal encryption enables efficient zero-knowledge proofs of these requirements, and the Fiat-Shamir
heuristic can be used to make these proofs non-interactive. Chaum-Pedersen proofs are used to demonstrate that an
encryption is that of a specified value, and these are combined with the Cramer-Damg˚ ard-Schoenmakers technique to show
that an encryption is that of one of a specified set of values – particularly that a value is an encryption of either
zero or one. The encryptions of selections in a contest are homomorphically combined, and the result is shown to be an
encryption of that contest’s selection limit – again using a Chaum-Pedersen
\cite[13]{eg-spec}
the result is shown to be an encryption of that contest’s selection limit – again using a Chaum-Pedersen proof.
\cite[13]{eg-spec}
The “random” nonces used for the ElGamal encryption of the ballot nonces are derived from a single 256-bit master nonce
RB for each ballot.
\cite[13]{eg-spec}

\subsection{Tracking codes}
Upon completion of the encryption of each ballot Bi, a tracking code Hi is prepared for each voter. The code is a
SHA-256 hash
\cite[17]{eg-spec}
previous tracking code. When the previous tracking code Hi is included, the chain is
The benefit of a chain is that it makes it more difficult for a malicious insider to selectively delete ballots and
tracking codes after an election without detection.
\cite[17]{eg-spec}

Once in possession of a tracking code (and never before), a voter is afforded an option to either cast the associated
ballot or spoil it and restart the ballot preparation process. The precise mechanism for voters to make these selections
may vary depending upon the instantiation, but this choice would ordinarily be made immediately after a voter is
presented with the tracking code, and the status of the ballot would be undetermined until the decision is made. It is
possible, for instance, for a voter to make the decision directly on the voting device, or a voter may instead be
afforded an option to deposit the ballot in a receptacle
\cite[17]{eg-spec}

\subsection{Ballot Aggregation}
At the conclusion of voting, all of the ballot encryptions are published in the election record together with the proofs
that the ballots are well-formed. Additionally, all of the encryptions of each option are homomorphically combined to
form an encryption of the total number of times that option was selected. The encryptions (αi, βi) of each individual
option are combined by forming the product (A, B) = (� i αi mod p, � i βi mod p). This aggregate encryption (A, B),
which represents an encryption of the tally of that option, is published in the election record for each option.
\cite[18]{eg-spec}

\subsection{Decryption when all guardians are present}
If all guardians are present and have posted suitable proofs, the next step is to publish the value
\cite[19]{eg-spec}.
one or more of the election guardians are not available for decryption, any k available guardians can use the
information they have to reconstruct the partial decryptions for missing guardians
\cite[19]{eg-spec}

\subsection{Decryption of spoiled ballots}
Every ballot spoiled in an election is individually verifiably decrypted in exactly the same way that the aggregate
ballot of tallies is decrypted.
\cite[21]{eg-spec}

The exponential ElGamal used to encrypt votes is defined by a 4096-bit prime p and a 256-bit prime q which divides (p −
1). We use r = (p − 1)/q to denote the cofactor of q, and a
\cite[21]{eg-spec}

\subsection{The Election Record}
The record of an election should be a full accounting of all of the election artifacts. Specifically, it should contain
the following. Date and location of an election. The ballot coding file. The baseline parameters:
\cite[24]{eg-spec}

The base hash value Q computed from the above. The commitments from each election guardian to each of their polynomial
coefficients. The proofs from each guardian of possession of each of the associated coefficients. The election public
key. The extended
\cite[24]{eg-spec}

Every encrypted ballot prepared in the election (whether cast or spoiled): – All of the encrypted options on each ballot
(including “placeholder” options), – the proofs that each such value is an encryption of either zero or one, – the
selection limit for each contest, – the proof that the number of selections made matches the selection limit, – the
device information for the device that encrypted the ballot, – the date and time of the ballot encryption, – the tracker
code produced for the ballot. The decryption of each spoiled ballot: – The selections made on the ballot, – the
cleartext representation of the selections, – partial decryptions by each guardian of each option, – proofs of each
partial decryption, – shares of each missing partial decryption (if any), – proofs of shares of each missing partial
decryption (if any), – Lagrange coefficients used for replacement of any missing partial decryptions (if any). Tallies
\cite[24]{eg-spec}


Lagrange coefficients used for replacement of any missing partial decryptions (if any). Tallies of each option in an
election: – The encrypted tally of each option, – full decryptions of each encrypted tally, – cleartext representations
of each tally, – partial decryptions by each guardian of each tally, – proofs of partial decryption of each tally, –
shares of each missing partial decryption (if any), – proofs of shares of each missing partial decryption (if any), –
Lagrange coefficients used for replacement of any missing partial decryptions (if any). Ordered lists of the ballots
encrypted by each device. An election record should be digitally signed by election administrators together with the
date of 24
\cite[24]{eg-spec}


\section{Esp32}

idf.py -p PORT flash monitor

ESP-IDF is the IoT development framework for ESP series SoCs provided by Espressif, including:


\begin{itemize}

    \item
    A series of libraries and header files, providing core components required for building software projects based on
    ESP SoC;



    \item Common tools and functions used during the development and
    manufacturing processes, e.g., build, flashing, debugging, measurement
    and etc.


\end{itemize}



Im Grundlagenkapitel werden die zum Verständnis des Hauptteils nötigen Grundlagen ausgearbeitet.
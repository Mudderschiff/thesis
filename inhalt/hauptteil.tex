\chapter{Hauptteil/Main Part}
Bachelor- und Masterarbeiten können sowohl in deutsch als auch in englisch geschrieben werden. 
Die sprachliche Ausarbeitung wird bewertet, was bei der Wahl der Sprache berücksichtigt werden sollte. 
Im folgenden werden ein paar Hinweise zur Ausarbeitung  mit \LaTeX\ gegeben. 

\section{Unterkapitel}
\subsection{Dritte Gliederungsebene}
Falls in einem Kapitel mehrere Gliederungsebenen verwendet werden sollte darauf geachtet werden, dass mindestens drei Punkte pro ebene existieren. 

\begin{table}[h]
	\centering
	\begin{tabular}{|l|l|l|}
		\hline
		1&2&3\\\hline
		4&5&6\\\hline
	\end{tabular}
	\caption{Beispieltabelle}
	\label{Tab:Beispieltabelle}
\end{table}

Hier wird die Beispieltabelle~\ref{Tab:Beispieltabelle} referenziert.

\begin{figure}[h] %in den eckigen Klammern wird angegeben, wo das Bild erscheinen soll:
	%h = here, t = top, b = bottom, p = page (eigene Seite für Bild/er)
	%TeX versucht es in dieser Reihenfolge schön hinzukriegen
	%wenn das erste gut aussieht, wird das genommen, sonst das zweite usw.
	\centering
	\includegraphics[scale=.25]{abbildungen/bild1}
	\caption[Bildunterschrift mit Quellenangabe]{Bildunterschrift mit Quellenangabe \cite{ungerer2013parmerasa}}
	\label{Fig:Bildbezeichnung}
\end{figure}

Hier wird die Beispielbild~\ref{Fig:Bildbezeichnung} referenziert.

\begin{equation} \label{eq:Beispielformel}
	\sum_{x=0}^{10}x=55
\end{equation}

Hier wird die Beispielformel~\ref{eq:Beispielformel} referenziert.

\textit{kursiv}, \textbf{fett}, \underline{unterstrichen}

Abkürzungen müssen im Abkürzungsverzeichnis angelegt werden.
Erste Verwendung einer \ac{ABK} jede weitere Verwendung der \ac{ABK}.

%Befehl um sämtliche Literatur im Literaturverzeichnis aufzuführen
\nocite{*}


\chapter{Hauptteil/Main Part}

\subsubsection{Key Generation}
Prior to the start of voting, trustees participate in a process wherein they generate public keys to be used in the
election. These trustees are called "Guardians" in ElectionGuard. Members of a canvassing board could serve as
guardians \cite[2]{eg-spec}.

Each guardian generates its own public-private key pair. These public keys will be combined to form a single public
key which will be used to encrypt individual ballots. They are also used individually by guardians to exchange
information about their private keys so that the election record can be produced after voting or auditing is
complete – even if not all guardians are available at that time.


The key generation ceremony begins with each guardian publishing its public keys together with proofs of knowledge
of the associated private keys. Once all public keys are published, each guardian uses each other guardian’s public
auxiliary key to encrypt shares of its own private keys. Finally, each guardian decrypts the shares it receives from
other guardians and checks them for consistency. If the received shares verify, the receiving guardian announces its
completion. If any shares fail to verify, the receiving guardian challenges the sender. In this case, the sender is
obliged to reveal the shares it sent. If it does so and the shares verify, the ceremony concludes and the election
proceeds. If a challenged guardian fails to produce key shares that verify, that guardian is removed and the key
generation ceremony restarts with a replacement guardian. \cite[2]{eg-spec}

Although it is preferred to generate new keys for each election, it is permissible to use the same
keys for multiple elections so long as the set of guardians remains the same. A complete new set of keys must
be generated if even a single guardian is replaced.\cite[3]{eg-spec}

5As will be seen below, the actual public key used to encrypt votes will be a combination of separately generated public
keys. So, no entity will ever be in possession of a private key that can be used to decrypt
-\cite[5]{eg-spec}

3.2 Key Generation Before an election, the number of guardians (n) is fixed together with a quorum value (k) that
describes the number of guardians necessary to decrypt tallies and produce election verification data. The values n and
k are integers subject to the constraint that $1 \leq k \leq n$. Canvassing board members can often serve the role of election
guardians, and typical values for n and k could be 5 and 3 – indicating that 3 of 5 canvassing board members must
cooperate to produce the artifacts that enable election verification. The reason for not setting the quorum value k too
low is that it will also be possible for k guardians to decrypt individual ballots.
\cite[8]{eg-spec}


\subsection{Key Generation}
of an election is a public ballot coding file. This file lists all the contests in an election, the number of selections
allowed for each contest, and the options for each contest together with associations between each option and its
representation on a virtual ballot. It is assumed that each contest in the ballot coding file has a unique label and
that within each contest, each option also has a unique label. For
\cite[8]{eg-spec}

But the contents of this file are hashed together with the prime modulus (p), subgroup order (q), generator (g), number
of guardians (n), decryption quorum threshold value (k), date, and jurisdictional information to form a base hash code (
Q) which will be incorporated into every subsequent hash computation in the election.
\cite[8]{eg-spec}

\subsection{Decryption when all guardians are present}
If all guardians are present and have posted suitable proofs, the next step is to publish the value
\cite[19]{eg-spec}.
one or more of the election guardians are not available for decryption, any k available guardians can use the
information they have to reconstruct the partial decryptions for missing guardians
\cite[19]{eg-spec}

The base hash value Q computed from the above. The commitments from each election guardian to each of their polynomial
coefficients. The proofs from each guardian of possession of each of the associated coefficients. The election public
key. The extended
\cite[24]{eg-spec}


\subsection{Elgamal}
Public-key cryptography allows anyone to exchange secret messages with any partners, sign documents and use many other cryptographic applications \cite[22]{crypto}. Diffie-Hellman is the first published public key algorithm by Diffie and Hellman in 1976.\cite[94]. ElGamal is a generalisation of the Diffie-Hellman key exchange protocol.\cite[95]{crypto} ElGamal is suitable for encryption and signing compared to Diffie-Hellman which is only suitable for key exchange.\cite[78]{crypto}.

\subsubsection{SHA-256}
SHA-256 is a one-way hash function. One-way functions are functions that are easy to calculate, but whose inversion is very difficult or impossible to calculate \cite[100]{crypto}. One-way hash functions map plaintexts of any length to a hash value of a fixed length. \cite[100]{crypto}.









\subsection{Non-interactive zero-knowledge (NIZK)}
Non-interactive zero-knowledge (NIZK) proofs ElectionGuard provides numerous proofs about encryption keys, encrypted
ballots, and election tallies using the following four techniques \cite[6]{eg-spec}
\subsubsection{Schnorr proof}
1. A Schnorr proof7 allows the holder of an ElGamal
secret key s to interactively prove possession of s without revealing s. \cite[6]{eg-spec}.
\subsubsection{Chaum-Pedersen}
2. A Chaum-Pedersen proof8 allows an ElGamal
encryption to be interactively proven to decrypt to a particular value without revealing the nonce used for encryption
or the secret decryption key s. (This proof can be constructed with access to either the nonce used for encryption or
the secret decryption key.) \cite[6]{eg-spec}
\subsubsection{Cramer-Damg ard-Schoenmakers technique}
3. The Cramer-Damg ard-Schoenmakers technique9 enables a disjunction to be interactively
proven without revealing which disjunct is true.
\subsubsection{Fiat-Shamir heuristic}
4. The Fiat-Shamir heuristic10 allows interactive proofs to be
converted into non-interactive proofs. \cite[6]{eg-spec}

Using a combination of the above techniques, it is possible for ElectionGuard to
demonstrate that keys are properly chosen, that ballots are properly formed, and that decryptions match claimed values.
\cite[6]{eg-spec}

Threshold Encryption
Additive Homomorphic encryption
\section{asw}

Since ElectionGuards original specification in 2019, there have been several implementations of ElectionGuard that have been used in various applications \cite[eg-paper]. The current roadmap of ElectionGuard targets a c++ implementation of the ElectionGuard 2.0 specification. [https://www.electionguard.vote/overview/Roadmap/]. Earlier implementations include a Python reference implementation of the ElectionGuard 1.0 specification and an encryption engine in C++ with a C wrapper.

ESP32 development support development of applications in C, C++ and Micropython. [source]. Initialy, we could try to port the encryption engine written in c++ over to ESP32. This would allow us to use the existing codebase and focus on the integration of the encryption component with the hardware.
The encryption engine would be responsible for generating an elgamal keypair and the subsequent exchange of cryptographic proofs and cryptographic keys. 

The modular exponentiation at the heart of most ElectionGuard operations imposes the highest computaitional cost among all computations and is the limiting factor in any performance analysis. Using fast libraries for modular arithmetic is crucial to achieve good performance so that the latency due to Key generation and ZK proof generation doesn not impact usability. [source]

The c++ implementation uses Microsoft's HACL* - a performant C implementation of a wide variety of cryptographic primitives which have been formally verified for correctness [source.]. For performance reasons the implementation of the c++ encryption engine  uses pre-computed tables to make encryption substantially faster. This is possible because most exponentiations in ElectionGuard have fixed base, either the generator g or the election public key K. The pre-computed tables contain certain powers of these bases. The Python reference implementation uses a more straightforwar approach by using GnuMP. [source].

The ESP32 is equipped with hardware accelerators of general algorithms such as SHA and RSA and it also supports independent arithmetic, such as Big Integer Multiplication and Big Integer Modular Multiplication. [esp tech reference,4.1.19]. The hardware accelerators greatly improve operation speed and reduce software complexity. 




\begin{table}[ht]
	\centering
	\begin{tabular}{|c|c|c|c|}
		\hline
		&LCD&Board&Description\\\hline
		1&VCC&3.3V\\\hline
		2&GND&GND\\\hline
		3&GND&GND\\\hline
		4&NC \\\hline
		5&NC \\\hline
		6&NC \\\hline
		7&CLK&D14&SPI-CLK\\\hline
		8&SDA&D13&SPI-MOSI\\\hline
		9&RS&D34&Any GPIO PIN\\\hline
		10&RST&D35&Any GPIO PIN\\\hline
		11&CS&D15&SPI-SS\\\hline
	\end{tabular}
	\caption{PINOUT LCD}
	\label{Tab:PINOUT_LCD}
\end{table}

Bachelor- und Masterarbeiten können sowohl in deutsch als auch in englisch geschrieben werden. 
Die sprachliche Ausarbeitung wird bewertet, was bei der Wahl der Sprache berücksichtigt werden sollte. 
Im folgenden werden ein paar Hinweise zur Ausarbeitung  mit \LaTeX\ gegeben.


\section{Unterkapitel}
\subsection{Dritte Gliederungsebene}
Falls in einem Kapitel mehrere Gliederungsebenen verwendet werden sollte darauf geachtet werden, dass mindestens drei Punkte pro ebene existieren. 

\begin{table}[h]
	\centering
	\begin{tabular}{|l|l|l|}
		\hline
		1&2&3\\\hline
		4&5&6\\\hline
	\end{tabular}
	\caption{Beispieltabelle}
	\label{Tab:Beispieltabelle}
\end{table}

Hier wird die Beispieltabelle~\ref{Tab:Beispieltabelle} referenziert.

\begin{figure}[h] %in den eckigen Klammern wird angegeben, wo das Bild erscheinen soll:
	%h = here, t = top, b = bottom, p = page (eigene Seite für Bild/er)
	%TeX versucht es in dieser Reihenfolge schön hinzukriegen
	%wenn das erste gut aussieht, wird das genommen, sonst das zweite usw.
	\centering
	\includegraphics[scale=.25]{abbildungen/bild1}
	\caption[Bildunterschrift mit Quellenangabe]{Bildunterschrift mit Quellenangabe \cite{lcd}}
	\label{Fig:Bildbezeichnung}
\end{figure}

Hier wird die Beispielbild~\ref{Fig:Bildbezeichnung} referenziert.

\begin{equation} \label{eq:Beispielformel}
	\sum_{x=0}^{10}x=55
\end{equation}

Hier wird die Beispielformel~\ref{eq:Beispielformel} referenziert.

\textit{kursiv}, \textbf{fett}, \underline{unterstrichen}

Abkürzungen müssen im Abkürzungsverzeichnis angelegt werden.
Erste Verwendung einer \ac{ABK} jede weitere Verwendung der \ac{ABK}.

%Befehl um sämtliche Literatur im Literaturverzeichnis aufzuführen
\nocite{*}

